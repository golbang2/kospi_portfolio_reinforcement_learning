\documentclass[a4paper]{article}

\usepackage[nojosa]{hangul}
%\usepackage{jkdiss_eng}%영문논문인 경우
\usepackage{kdiss_tabfig}%
\usepackage{amsmath,graphicx,chicago}
\usepackage{amssymb}
\usepackage{multirow}
\usepackage{array}
\usepackage{hhline}

%% ==================== 여기는 고치지 마십시오. =======================
%\submit{draft}
\submit{final}
\thefirstpage{1}
\receive{0000}{0}{1}
\revise{0000}{0}{0}%수정일자
\accept{0000}{0}{0}%채택일자
\volumn{00}{0}
%% ====================================================================
%\runningtitleauthor{Development of failure reporting analysis and corrective action system}{Hong Yeon Woong}
\heading{Article for JKDISS}{author1 $\cdot$ author2}

\begin{document}

\title{유가증권 시장에서 강화학습을 이용한 종목선택과 포트폴리오 최적화}% {}\footnote{사사 2.} }

\author{
김태윤\footnote{(61186) 광주광역시 북구 용봉로 77 자연대 1호관 238, 전남대학교 통계학과, 석사과정. E-mail: 196350@jnu.ac.kr} $\cdot$
고봉균\footnote{(61186) 광주광역시 북구 용봉로 77 자연대 1호관 , 전남대학교 경영학부, 교수. \\ E-mail: nadozzzzz@choizzz.ac.kr} 
}
\address{
${}^{1}${전남대학교 통계학과} $\cdot$ ${}^{2}${전남대학교 통계학}
}

\accepted
\begin{abstract}
주식 투자와 자산 관리에서 포트폴리오 분배와 최적화는 위험을 관리하고 수익률을 극대화하기 위해 필수적인 부분으로 금융분야에서 해결해야 할 전통적인 문제였다. 한편 최근 딥러닝이 많은 연구가 이루어지고 큰 성과를 이루었고 그와 함께 강화학습 또한 큰 발전을 이루고 있다. 이에 따라 최근 포트폴리오 관리에 강화학습 방법론을 적용하려는 시도가 이루어졌지만 연구의 대부분은 거래 규모가 큰 암호화폐에 한정되어 이루어 진 것이 대부분이다. 본 논문에서는 유가증권시장의 상위 종목 중 대표성이 높은 종목으로 선정되는 KOSPI200을 구성하는 종목 중 투자 대상 주식을 선정하는 selector 네트워크와 선정된 주식을 배분하는 allocator 네트워크 두가지를 통해 포트폴리오를 구성하는 신경망을 강화학습을 통해 구현하였다.
\end{abstract}

\begin{keywords}{강화학습, 포트폴리오 이론, 딥러닝,코스피}
\end{keywords}

\section{서론}
내용이 여기에 옮.

\section{FRACAS 개요}
내용이 여기에 옮.

\begin{table}[!ht]
\label{tbl:data}
\begin{center}
{\scriptsize
\tablcap{ 표의 제목}

\begin{tabular}{cccccc} \hline \hline 
\multicolumn{1}{c}{열제목1} & \multicolumn{1}{c}{열제목2} & \multicolumn{1}{c}{열제목3} & \multicolumn{1}{c}{열제목4} & \multicolumn{1}{c}{열제목5} & \multicolumn{1}{c}{열제목6} \\ \cline{1-6}
\multicolumn{1}{c}{1} & \multicolumn{1}{c}{123} & \multicolumn{1}{c}{88} & \multicolumn{1}{c}{33} & \multicolumn{1}{c}{aa} & \multicolumn{1}{c}{999} \\ 
\hline 
\end{tabular}
}
\end{center}
\end{table}


내용이 여기에 옮.
\begin{equation*}
h (y) = \sin \biggl (P_1 (\beta;y) + R_2 (\gamma;y) \biggr),
\end{equation*}
 
 내용이 여기에 옮.
  
 \begin{equation} \label{model}
h (x) = \log \left ( \int_{i,\ j} \alpha_{[i][j]} x_{ij} x_{i (j+1)} \ + \ \beta_{[i][j]} y_{ij} mu_{ (i+1)j} \right),
\end{equation}

내용이 여기에 옮.
 
 \begin{figure}[!ht]
	\centering
	\includegraphics[width=0.6\textwidth]{figure1_1}  
	\begin{center}
	\figcap{그림 제목}  
	\end{center}
	\label{fig1:data}
\end{figure}

 내용이 여기에 옮. 

 
\theorem 정리 내용
\proof 증명이 여기에 옮.
\endproof

\begin{references}
\bibitem[author (2008)]{key}Kang, S. M. and Kim, J. H. (2008). Statistical model of effective impact speed based on vehicle damages in case of rear-end collisions.{\em Journal of the\/}{\em Korean Data \& Information Science Society\/}, {\bf{19}}, 463-473. 
\end{references}

\appendix
\section{절 제목}{}
내용이 여기에 옮.

\theorem 정리의 내용이 여기에 옮.
\proof 정리의 증명이 여기에.
\endproof

\etitle{Article title\footnote{This reserach bla bla bla....}}

\eauthor{
{Author First}\footnote{{Professor, Department of Statistics, Gyeongbuk 712-749, Korea.} \\E-mail: zzzzaong@azzzz.ac.kr }
 $\cdot$ {Author Second}\footnote{{Professor, Department of Management, Seoul 100-100, Korea.} \\E-mail: zzzzaong@bzzzz.ac.kr }}
\eaddress{
${}^{}${Department of Statistics, Azzzz University}\\${}^{}${Department of Management, Bzzzz University}
}

\eaccepted
\begin{eabstract}
Some sentences come here.
\end{eabstract}

\ekeywords{ key1, key2. }

\end{document}